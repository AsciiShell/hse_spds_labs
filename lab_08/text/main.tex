%!TEX TS-program = xelatex

% Шаблон документа LaTeX создан в 2018 году
% Алексеем Подчезерцевым
% В качестве исходных использованы шаблоны
% 	Данилом Фёдоровых (danil@fedorovykh.ru) 
%		https://www.writelatex.com/coursera/latex/5.2.2
%	LaTeX-шаблон для русской кандидатской диссертации и её автореферата.
%		https://github.com/AndreyAkinshin/Russian-Phd-LaTeX-Dissertation-Template

\documentclass[a4paper,14pt]{article}

\input{data/preambular.tex}
\begin{document} % конец преамбулы, начало документа
	\input{data/title.tex}
	\tableofcontents
	\pagebreak
	\section{Задание}
	
	\begin{enumerate}
		
		\item Изучить раздел 5.2 Audio Recording and Playing в \_DE1-SoC\_User\_manual.pdf.
		
		\item Модифицировать файл DE1\_SoC\_Audio.bat под выполнение на вашем компьютере.
		
		\item Запустить проект и продемонстрировать его работу.
		
		\item Запустить и загрузить в SOPC аппаратную часть проекта. Отобразить в отчете, структуру NIOS ядра, описать какие компоненты к ядру подключены и для чего.
		
		\item Запустить Nios II Software Build Tools for Eclipse. Отобразить в отчете схему организации программной части проекта. Найти точку входа в проект. Уметь объяснить назначение используемых функций.
		
		\item Изучить раздел 5.3 Karaoke Machine в \_DE1-SoC\_User\_manual.pdf.
		
		\item Модифицировать файл DE1\_SoC\_i2sound.bat под выполнение на вашем компьютере.
		
		\item Запустить проект и продемонстрировать его работу.
		
		\item Описать отличия проекта SPDS\_Lab\_8\_DE1\_SoC\_i2sound от \\ SPDS\_Lab\_8\_DE1\_SoC\_Audio в их реализации.
		
		\item Модифицировать пример из раздела 5.2 так, чтобы с помощью кнопок SW6 – SW9 можно было регулировать громкость воспроизведения. А кнопкой 1 включать пищалку.
		
	\end{enumerate}

	%{\small \VerbatimInput{../03_syn_pow_5_single_cycle_always/pow_5_single_cycle_always.v}}
	
	\section{Выполнение работы}
	
	На рис. \ref{fig:shemeqsys} приведена блок схема, показывающая связь между компонентами, подключенными к ядру.
	На рис. \ref{fig:qsys} скриншот связи между компонентами из программы Platform Designer.
	
	\begin{enumerate}
		
		\item Nios II -- процессорное ядро;
		
		\item JTAG UART -- компонент, позволяющий взаимодействовать плате и ПК;
		
		\item PLL -- компонент управляющий записью в память памятью;
		
		\item SDRAM Controller -- компонент управляющий памятью;
		
		\item SDRAM0 -- память, на которой храниться записанный голос;
		
		\item On Chip Memory -- память с программой;
		 
		\item AUDIO Controller -- компонент для взаимодействия с кодеком;
		
		\item AUDIO -- кодек преобразующий сигнал с линейного входа в цифровой сигнал и наоборот;
		
		\item PIO Controller -- компонент, позволяющий общаться с периферией платы;
		
		\item System Interconnect Fabric -- шина данных.
		
	\end{enumerate}
	
	\begin{figure}[H]
		\centering
		\includegraphics[width=0.7\linewidth]{images/sheme_qsys}
		\caption{Блок диаграмма связи между компонентами}
		\label{fig:shemeqsys}
	\end{figure}

	\begin{figure}[H]
		\centering
		\includegraphics[width=0.55\linewidth]{images/QSYS1}
		\caption{Все компоненты периферии}
		\label{fig:qsys}
	\end{figure}

	Программная часть состоит из нескольких файлов, запускается файл $main.c$, в нем точкой входа является функция main.
	головные файлы остальных подгружаемых библиотек: $my\_includes.h$, $AUDIO.h$, $LED.h$, $SEG7.h$ и $<math.h>$.
	
	На рис. \ref{fig:shemeqsys2} представлена блок схема для Karaoke machine.
	
	\begin{figure}[H]
		\centering
		\includegraphics[width=0.7\linewidth]{images/sheme_qsys_2}
		\caption{Блок диаграмма для Karaoke machine}
		\label{fig:shemeqsys2}
	\end{figure}
	
	Проект i2sound реализован на самом железе платы, включая кодек.
	Данные с линейных входов поступают сразу к нему, складываются и в преобразованном виде отправляются на линейный выход.
	При этом к кодеку подключены различные элементы управления.

	Проект Audio реализован на готовом ядре процессора.
	Кодек реализован в программном виде и в виде компонента присоединен через шину к процессору.
	
	При этом в проектах используются одинаковый интерфейс i2c.	
	

	\section{Самостоятельная работа}
	
	Модифицировать пример из раздела 5.2 так, чтобы с помощью кнопок SW6 – SW9 можно было регулировать громкость воспроизведения. 
	А кнопкой 1 включать пищалку.
	
	Пример был модифицирован.

	\section{Выводы по работе}
	
	В ходе работы было получено представление о звуке в цифровых устройствах.
	Был получен опыт в написании и анализе кода на языке С.
	Был получен опыт генерации звука при помощи ПЛИС и языка С.
	Было изучено хранение звука в памяти ПЛИС.
	Был получен опыт в обработке звука пр помощи компонентов в $Platform Designer$.
	Был изучен способ программного повышения громкости в языке С.
	Итоговый проект был собран и загружен на плату.

	
	\newpage 
	\renewcommand{\refname}{{\normalsize Список использованных источников}} 
	\centering 
	\begin{thebibliography}{9} 
		\addcontentsline{toc}{section}{\refname} 
		\bibitem{Harris} Хэррис Д. М., Хэррис С. Л. Цифровая схемотехника и архитектура компьютера. – 2015.
	\end{thebibliography}
	
\end{document} % конец документа
