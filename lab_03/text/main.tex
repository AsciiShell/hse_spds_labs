%!TEX TS-program = xelatex

% Шаблон документа LaTeX создан в 2018 году
% Алексеем Подчезерцевым
% В качестве исходных использованы шаблоны
% 	Данилом Фёдоровых (danil@fedorovykh.ru) 
%		https://www.writelatex.com/coursera/latex/5.2.2
%	LaTeX-шаблон для русской кандидатской диссертации и её автореферата.
%		https://github.com/AndreyAkinshin/Russian-Phd-LaTeX-Dissertation-Template

\documentclass[a4paper,14pt]{article}

\input{data/preambular.tex}
\begin{document} % конец преамбулы, начало документа
	\input{data/title.tex}
	\tableofcontents
	\pagebreak
	\section{Задание}
	
	\begin{enumerate}
		\item Создать модуль RAM памяти с двумя входами/выходами

		\item Добавить к процессору в тракт передачи данных блок памяти на 16-ть 32-х битных слов с двумя входами/выходами
		
		\item Добавить команды LW и SW в тракт передачи данных

		\item разработать программу на ассемблере, которая: 6) Инвертирует все четные числа в памяти.
		
		\item Синтезировать проект, провести моделирование и прототипирование процессора и выполняемой программы. 
		
		\item Сравнить данный вариант процессора с версией https://github.com/MIPSfpga/schoolMIPS/tree/01\_mmio/. В чем отличия?

		\item Перейти в ветку проекта schoolMIPS 03\_pipeline: https://github.com/MIPSfpga/schoolMIPS/tree/03\_pipeline.
		
		\item Скачать новую версию процессора и выполнить на вашей плате (или DE10-Lite) программы 00\_counter, 01\_fibonacci, 02\_sqrt, 03\_ram.
		
		\item Выполнить программы: 06\_hz\_forward, 07\_hz\_stall, 08\_hz\_branch
	\end{enumerate}

	%{\small \VerbatimInput{../03_syn_pow_5_single_cycle_always/pow_5_single_cycle_always.v}}
	
	\section{Выполнение работы}
	
	
	\subsection{Создание модуля RAM}

	Был создан модуль RAM памяти с двумя входами/выходами на языке Verilog.
	Для этого использовалась функция Insert Template в окне редактора кода.
	
	\section{Самостоятельная работа}
	
	
	\subsection{Доработка процессора}
	
	
	За основу была взята программа с первой лабораторной работы, в которой во время первой лабораторной работы были дополнительно реализованны команды sllv и nor.
	
	Изменения в процессоре:
	
	\begin{enumerate}
		\item Были добавлены команды sw и lw, по аналогии с второй лабораторной работой.
		
		\item Был добавлен модуль matrix, который содержит в себе модуль RAM.
		
		\item Вход и выход для просмотра состояния регистров в процессоре был переделан на просмотр содержимого в RAM.
	\end{enumerate}
	
	Отличие этого процессора от 01\_mmio заключается в том, что этот процессор не поддерживает работу периферийных устройств.
	
	\subsection{Разработка программы, которая инвертирует все четные числа в памяти}
	
	Код программы:
	
	{\small \VerbatimInput{../program/16_invert_even/main.S}}
	
	Результат моделирования в Icarus verilog (рис. \ref{fig:wvf16}).
	
	\begin{figure}[H]
		\centering
		\includegraphics[width=0.7\linewidth]{images/wvf_16}
		\caption{Моделирование программы в Icarus verilog}
		\label{fig:wvf16}
	\end{figure}	

	\subsection{Моделирование программ на процессоре из ветки 03\_pipeline}
	
	Программы 00\_counter, 01\_fibonacci, 02\_sqrt, 03\_ram продолжают работать на данной версии процессора.
	
	
	\subsubsection{Выполнение программы 06\_hz\_forward}
	
	Код программы:
	
	{\small \VerbatimInput{../program/06_hz_forward/main.S}}
	
	Лог выполнения программы:
	
	{\small \VerbatimInput{logs/06_hz_forward.txt}}
	
	Результат моделирования в Icarus verilog (рис. \ref{fig:wvf06}).
	
	\begin{figure}[H]
		\centering
		\includegraphics[width=0.7\linewidth]{images/wvf_06}
		\caption{Моделирование программы в Icarus verilog}
		\label{fig:wvf06}
	\end{figure}

	На вэйвформе можно заметить, что данные в регистрах появляются спустя несколько тактов после попадания инструкции в процессор.
	
	
	\subsubsection{Выполнение программы 07\_hz\_stall}
	
	Код программы:
	
	{\small \VerbatimInput{../program/07_hz_stall/main.S}}
	
	Лог выполнения программы:
	
	{\small \VerbatimInput{logs/07_hz_stall.txt}}
	
	Результат моделирования в Icarus verilog (рис. \ref{fig:wvf07}).
	
	\begin{figure}[H]
		\centering
		\includegraphics[width=0.7\linewidth]{images/wvf_07}
		\caption{Моделирование программы в Icarus verilog}
		\label{fig:wvf07}
	\end{figure}

	В данной программе сначала число записывается в оперативную память, потом считывается из оперативной памяти.
	При этом арифметическая операция над регистром, с которы считывается число из RAM, не производится.
	Это можно заметить в 10-11 строках лога выполнения программы.
	
	\subsubsection{Выполнение программы 08\_hz\_branch}
	
	Код программы:
	
	{\small \VerbatimInput{../program/08_hz_branch/main.S}}
	
	Лог выполнения программы:
	
	{\small \VerbatimInput{logs/08_hz_branch.txt}}
	
	Результат моделирования в Icarus verilog (рис. \ref{fig:wvf08}).
	
	\begin{figure}[H]
		\centering
		\includegraphics[width=0.7\linewidth]{images/wvf_08}
		\caption{Моделирование программы в Icarus verilog}
		\label{fig:wvf08}
	\end{figure}

	В данной программе демонстрируется то, что после инструкции ветвления не происходит выполнение инструкции следующей за текущей.
	Это можно наблюдать в (7-8, 12-13, 15-17) строках лога. 
	
	\section{Выводы по работе}
	
	В ходе работы получен опыт в написании кода на ассемблере MIPS.
	Был получен опыт чтения машинного кода.
	Был доработан процессор MIPS, к нему была добавлена оперативная память.
	Изучены и добавлены дополнительные команды процессора MIPS.
	Было произведено моделирование в программе Icarus Verilog.
	Итоговый проект был собран и загружен на плату.
	Был рассмотрен процессор с конвейерной архитектурой.
	
	\newpage 
	\renewcommand{\refname}{{\normalsize Список использованных источников}} 
	\centering 
	\begin{thebibliography}{9} 
		\addcontentsline{toc}{section}{\refname} 
		\bibitem{Harris} Хэррис Д. М., Хэррис С. Л. Цифровая схемотехника и архитектура компьютера. – 2015.
	\end{thebibliography}
	
\end{document} % конец документа
