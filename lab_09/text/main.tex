%!TEX TS-program = xelatex

% Шаблон документа LaTeX создан в 2018 году
% Алексеем Подчезерцевым
% В качестве исходных использованы шаблоны
% 	Данилом Фёдоровых (danil@fedorovykh.ru) 
%		https://www.writelatex.com/coursera/latex/5.2.2
%	LaTeX-шаблон для русской кандидатской диссертации и её автореферата.
%		https://github.com/AndreyAkinshin/Russian-Phd-LaTeX-Dissertation-Template

\documentclass[a4paper,14pt]{article}

\input{data/preambular.tex}
\begin{document} % конец преамбулы, начало документа
	\input{data/title.tex}
	\tableofcontents
	\pagebreak
	\section{Задание}
	
	\begin{enumerate}
		
		\item Изучить раздел 5.4 SDRAM Test in Nios II в \_DE1-SoC\_User\_manual.pdf;
		
		\item Запустить проект и продемонстрировать его работу;
		
		\item Изучить раздел 5.5 SDRAM Test in Verilog в \_DE1-SoC\_User\_manual.pdf;
		
		\item Запустить проект и продемонстрировать его работу;
		
		\item Описать отличия проекта SPDS\_Lab\_9\_DE1\_SoC\_SDRAM\_Nios\_Test, SPDS\_Lab\_9\_DE1\_SoC\_SDRAM\_RTL\_Test в их реализации;
		
		\item Изучить раздел 5.7 PS/2 Mouse Demonstration в \_DE1-SoC\_User\_manual.pdf;
		
		\item Запустить проект и продемонстрировать его работу;
		
		\item Привести описание проекта и его структуры, провести моделирование и включить в отчет вейвформы (Functional и Timing), демонстрирующие работу устройства;
		
		\item
		
		\item
		
	\end{enumerate}
	\pagebreak
	%{\small \VerbatimInput{../03_syn_pow_5_single_cycle_always/pow_5_single_cycle_always.v}}
	
	\section{Выполнение работы}
	
	\subsection{SDRAM Test in Nios II} 
	
	На рис. \ref{fig:niossh1} приведена структура построения NIOS ядра из проекта 5.4 SDRAM Test in Nios II.
	
	\begin{figure}[H]
		\centering
		\includegraphics[width=0.7\linewidth]{images/nios_sh_1}
		\caption{Структура построения NIOS ядра}
		\label{fig:niossh1}
	\end{figure}
	
	Ниже приведены компоненты из NIOS ядра и их описания.
	
	\begin{itemize}
		
		\item clk\_50 -- генерация импульса с частотой 50 МГц;
		
		\item pll -- компонент для увеличения частоты и ее синхронизации;
		
		\item nios2 -- процессорное ядро;
		
		\item onchip\_memory -- память выделяемая на кристалле;
		
		\item sysid\_qsys -- идентификатор системы;
		
		\item timer -- таймер или счетчик;
		
		\item jtag\_uart -- модуль для взаимодействия с ПК;
		
		\item sdram -- оперативная память;
		
		\item key -- модуль для работы с кнопками на плате.
		
	\end{itemize}

	Программная часть состоит из двух проектов, проект самой прошивки и проект BSP (Board Support Package).
	BSP генерируется на на основе выбранных параметров для платы.
	Далее необходимо собрать проект с прошивкой, входной точкой является файл main.c, в котором подключаются все необходимые библиотеки для работы с платой.
	
	\subsection{SDRAM Test in Verilog} 
	
	В первом проекте (SDRAM Test in Nios II) использовалось NIOS ядро и вся работа по обработке данных производилась в этом синтезируемом NIOS ядре. При этом программа для ядра (прошивка) писалась на языке C.
	
	Во втором проекте (SDRAM Test in Verilog) синтезируются другие структуры, которые предназначены только для данной задачи. При этом плате не нужна никакая другая программа или прошивка. 
	
	Главное отличие между проектами в том, что в первом используется универсальный NIOS процессор, а во втором способе используются специализированные структуры.
	
	\subsection{5.7 PS/2 Mouse Demonstration} 
	
	

	\section{Самостоятельная работа}
	


	\section{Выводы по работе}
	
	В ходе работы было получено представление о звуке в цифровых устройствах.
	Был получен опыт в написании и анализе кода на языке С.
	Был получен опыт генерации звука при помощи ПЛИС и языка С.
	Было изучено хранение звука в памяти ПЛИС.
	Был получен опыт в обработке звука пр помощи компонентов в $Platform Designer$.
	Был изучен способ программного повышения громкости в языке С.
	Итоговый проект был собран и загружен на плату.

	
	\newpage 
	\renewcommand{\refname}{{\normalsize Список использованных источников}} 
	\centering 
	\begin{thebibliography}{9} 
		\addcontentsline{toc}{section}{\refname} 
		\bibitem{Harris} Хэррис Д. М., Хэррис С. Л. Цифровая схемотехника и архитектура компьютера. – 2015.
	\end{thebibliography}
	
\end{document} % конец документа
