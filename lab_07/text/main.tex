%!TEX TS-program = xelatex

% Шаблон документа LaTeX создан в 2018 году
% Алексеем Подчезерцевым
% В качестве исходных использованы шаблоны
% 	Данилом Фёдоровых (danil@fedorovykh.ru) 
%		https://www.writelatex.com/coursera/latex/5.2.2
%	LaTeX-шаблон для русской кандидатской диссертации и её автореферата.
%		https://github.com/AndreyAkinshin/Russian-Phd-LaTeX-Dissertation-Template

\documentclass[a4paper,14pt]{article}


%%% Работа с русским языком
\usepackage[english,russian]{babel}   %% загружает пакет многоязыковой вёрстки
\usepackage{fontspec}      %% подготавливает загрузку шрифтов Open Type, True Type и др.
\defaultfontfeatures{Ligatures={TeX},Renderer=Basic}  %% свойства шрифтов по умолчанию
\setmainfont[Ligatures={TeX,Historic}]{Times New Roman} %% задаёт основной шрифт документа
\setsansfont{Comic Sans MS}                    %% задаёт шрифт без засечек
\setmonofont{Courier New}
\usepackage{indentfirst}
\frenchspacing

\renewcommand{\epsilon}{\ensuremath{\varepsilon}}
\renewcommand{\phi}{\ensuremath{\varphi}}
\renewcommand{\kappa}{\ensuremath{\varkappa}}
\renewcommand{\le}{\ensuremath{\leqslant}}
\renewcommand{\leq}{\ensuremath{\leqslant}}
\renewcommand{\ge}{\ensuremath{\geqslant}}
\renewcommand{\geq}{\ensuremath{\geqslant}}
\renewcommand{\emptyset}{\varnothing}

%%% Дополнительная работа с математикой
\usepackage{amsmath,amsfonts,amssymb,amsthm,mathtools} % AMS
\usepackage{icomma} % "Умная" запятая: $0,2$ --- число, $0, 2$ --- перечисление

%% Номера формул
%\mathtoolsset{showonlyrefs=true} % Показывать номера только у тех формул, на которые есть \eqref{} в тексте.
%\usepackage{leqno} % Нумерация формул слева	

%% Перенос знаков в формулах (по Львовскому)
\newcommand*{\hm}[1]{#1\nobreak\discretionary{}
	{\hbox{$\mathsurround=0pt #1$}}{}}

%%% Работа с картинками
\usepackage{graphicx}  % Для вставки рисунков
\graphicspath{{images/}}  % папки с картинками
\setlength\fboxsep{3pt} % Отступ рамки \fbox{} от рисунка
\setlength\fboxrule{1pt} % Толщина линий рамки \fbox{}
\usepackage{wrapfig} % Обтекание рисунков текстом

%%% Работа с таблицами
\usepackage{array,tabularx,tabulary,booktabs} % Дополнительная работа с таблицами
\usepackage{longtable}  % Длинные таблицы
\usepackage{multirow} % Слияние строк в таблице
\usepackage{float}% http://ctan.org/pkg/float

%%% Программирование
\usepackage{etoolbox} % логические операторы


%%% Страница
\usepackage{extsizes} % Возможность сделать 14-й шрифт
\usepackage{geometry} % Простой способ задавать поля
\geometry{top=20mm}
\geometry{bottom=20mm}
\geometry{left=20mm}
\geometry{right=10mm}
%
%\usepackage{fancyhdr} % Колонтитулы
% 	\pagestyle{fancy}
%\renewcommand{\headrulewidth}{0pt}  % Толщина линейки, отчеркивающей верхний колонтитул
% 	\lfoot{Нижний левый}
% 	\rfoot{Нижний правый}
% 	\rhead{Верхний правый}
% 	\chead{Верхний в центре}
% 	\lhead{Верхний левый}
%	\cfoot{Нижний в центре} % По умолчанию здесь номер страницы

\usepackage{setspace} % Интерлиньяж
\onehalfspacing % Интерлиньяж 1.5
%\doublespacing % Интерлиньяж 2
%\singlespacing % Интерлиньяж 1

\usepackage{lastpage} % Узнать, сколько всего страниц в документе.

\usepackage{soul} % Модификаторы начертания

\usepackage{hyperref}
\usepackage[usenames,dvipsnames,svgnames,table,rgb]{xcolor}
\hypersetup{				% Гиперссылки
	unicode=true,           % русские буквы в раздела PDF
	pdftitle={Заголовок},   % Заголовок
	pdfauthor={Автор},      % Автор
	pdfsubject={Тема},      % Тема
	pdfcreator={Создатель}, % Создатель
	pdfproducer={Производитель}, % Производитель
	pdfkeywords={keyword1} {key2} {key3}, % Ключевые слова
	colorlinks=true,       	% false: ссылки в рамках; true: цветные ссылки
	linkcolor=black,          % внутренние ссылки
	citecolor=black,        % на библиографию
	filecolor=magenta,      % на файлы
	urlcolor=black           % на URL
}
\makeatletter 
\def\@biblabel#1{#1. } 
\makeatother
\usepackage{cite} % Работа с библиографией
%\usepackage[superscript]{cite} % Ссылки в верхних индексах
%\usepackage[nocompress]{cite} % 
\usepackage{csquotes} % Еще инструменты для ссылок

\usepackage{multicol} % Несколько колонок

\usepackage{tikz} % Работа с графикой
\usepackage{pgfplots}
\usepackage{pgfplotstable}

% ГОСТ заголовки
\usepackage[font=small]{caption}
%\captionsetup[table]{justification=centering, labelsep = newline} % Таблицы по правобу краю
%\captionsetup[figure]{justification=centering} % Картинки по центру


\newcommand{\tablecaption}[1]{\addtocounter{table}{1}\small \begin{flushright}\tablename \ \thetable\end{flushright}%	
\begin{center}#1\end{center}}

\newcommand{\imref}[1]{рис.~\ref{#1}}

\usepackage{multirow}
\usepackage{spreadtab}
\newcolumntype{K}[1]{@{}>{\centering\arraybackslash}p{#1cm}@{}}


\usepackage{xparse}
\usepackage{fancyvrb}

\RecustomVerbatimCommand{\VerbatimInput}{VerbatimInput}
{
	fontsize=\footnotesize    
}

\usepackage{tocloft}
\renewcommand{\cftsecleader}{\cftdotfill{\cftdotsep}}
\begin{document} % конец преамбулы, начало документа
	\begin{titlepage}
	\begin{center}
 		ФЕДЕРАЛЬНОЕ  ГОСУДАРСТВЕННОЕ АВТОНОМНОЕ \\
		ОБРАЗОВАТЕЛЬНОЕ УЧРЕЖДЕНИЕ ВЫСШЕГО ОБРАЗОВАНИЯ\\
		«НАЦИОНАЛЬНЫЙ ИССЛЕДОВАТЕЛЬСКИЙ УНИВЕРСИТЕТ\\
		«ВЫСШАЯ ШКОЛА ЭКОНОМИКИ»
	\end{center}
	
	\begin{center}
		\textbf{Московский институт электроники и математики}
		
		\textbf{им. А.Н.Тихонова НИУ ВШЭ}
		
		\vspace{2ex}
		
		\textbf{Департамент компьютерной инженерии}
	\end{center}
	\vspace{1ex}	
	
	\begin{center}
		Курс «Системное проектирование цифровых устройств»
	\end{center}	
	
	
	\begin{center}
	\textbf{ОТЧЕТ\\
		ПО ЛАБОРАТОРНОЙ РАБОТЕ №5
	}
	\end{center}	

	\begin{center}
		Тема работы: «Обработка звука на ПЛИС»
	\end{center}

	\vspace{2ex}

	\begin{flushright}
		\textbf{Выполнили:}
		
		\vspace{2ex}
		
		Студенты группы БИВ174
		
		Бригада №5
		
		\vspace{2ex}
		
		Подчезерцев Алексей Евгеньевич
		
		Солодянкин Андрей Александрович
		\vspace{2ex}
		
		\textbf{Принял:}
		
		асс. МИЭМ НИУ ВШЭ
		
		Американов А.А.
		
	\end{flushright}

	\vfill
	\begin{center}
		Москва \the\year \, г.
	\end{center}
	
\end{titlepage}
\addtocounter{page}{1}
	\tableofcontents
	\pagebreak
	\section{Задание}
	
	\begin{enumerate}
		
		\item	Изучить раздел 5.1 DE1-SoC Factory Configuration в \_DE1-SoC\_User\_manual.pdf.
		
		\item	Модифицировать файл SPDS\_Lab\_7\_DE1\_SoC\_Default\\demo\_batch\\test.bat под выполение на вашем компьютере. 
		
		\item	Запустить проект и продемонстрировать его работу. 
			
		\item	Отобразить в отчете, каким образом генерируется изображение и передается на экран, в каком формате хранится рисунок.
		
		\item	Отобразить в отчете, как происходит генерация звука в примере проекта.
		
		\item	Отразить в отчете ответы на следующие вопросы: Что такое EPCS и для чего оно нужно? В чем отличия между форматами файлов *.jic и *.sof?
		
		\item	Вывести на 7-сегментный индикатор первые буквы имен участников бригады и номер бригады. Реализовать индикацию действий на светодиодной полосе. 

		\item	Модифицировать пример так, чтобы выводилось другое или существенно измененное изображение (изменен не только цвет). 
		
		
	\end{enumerate}

	%{\small \VerbatimInput{../03_syn_pow_5_single_cycle_always/pow_5_single_cycle_always.v}}
	
	\section{Выполнение работы}
	
	\subsection{Генерация изображения}
	
		Изображение хранится в виде 2 файлов.
		Первый файл $index\_logo.mif$ -- таблица цветов, второй $img\_data\_logo.mif$ таблица индексов.
		Индексы во втором файле это адреса в первом файле, где берется цвет пикселя для 3 цветовых каналов.
		В ПЛИС эти файлы загружаются при помощи ROM памяти.
		
		Данные изображения представляет из себя один сплошной поток адресов, чтобы разделить этот поток на строки используется $video_sync_generator$.
		В этом модуле в нужные моменты подаются сигналы о конце строки и конце изображения.
		
		После получения цвета пикселя в виде 24х битного слова оно делится на 3 канала и эти 3 значения отправляются на VGA интерфейс.
		
	\subsection{Генерация звука}
	
		Генерацию звука можно разделить на 2 этапа.
		На первом этапе задается частота на которой будет воспроизводиться звук ($VGA\_Audio\_0002.v$).
		Во время второго этапа задается амплитуда для поступающей частоты, на первый взгляд амплитуда сигнала похожа на синусоиду ($AUDIO\_DAC.v$).
		
		Принцип работы следующий, пин на который с определенной частотой поступает сигал с первого этапа. 
		Далее на этой частоте сигнал попадает на второй этап, где получает амплитуду.
		После этого сигнал отправляется в кодек, откуда и поступает на выход.
		
	\subsection{EPCS, sof, jic}
		
		EPCS - это ПЗУ на плате для конфигурации системы.
		Ядро устройства EPC разделено на два основных блока -- контроллер конфигурации и флэш-память. 
		Флэш - память используется для хранения конфигурационных данных для систем.
		Неиспользуемые части флэш-памяти можно использовать для хранения кода процессора или данных, доступ к которым можно получить с помощью внешнего интерфейса флэш-памяти после завершения конфигурации ПЛИС.
		
		Отличие jic файла от sof файла заключается в том, что jic фалом программируют через EPCS.  
	

	\section{Самостоятельная работа}
	
		Вывести на 7-сегментный индикатор первые буквы имен участников бригады и номер бригады. Реализовать индикацию действий на светодиодной полосе. 
	
		При нажатии key[0] на светодиодную ленту будут выводиться значения с кнопок.		
		При нажатии key[3] на семи сегментных индикаторах будут появится надпись: "Б5САПА".
		
		Модифицировать пример так, чтобы выводилось другое или существенно измененное изображение (изменен не только цвет). 
		
		Была скачана картинка того же размера, что и в примере.
		Далее при помощи python был создан файл с адресами цветов, аналогичный $img\_data\_logo.mif$.
		Цвета выбирались наиболее подходящие к имеющимся в наборе.
		

	\section{Выводы по работе}
	
	В ходе работы было получено представление о звуке и изображении в цифровых устройствах.
	Был получен опыт в написании и анализе кода на языке verilog.
	Был получен опыт генерации звука при помощи ПЛИС и языка verilog.
	Было изучено хранение изображений в памяти ПЛИС.
	Был получен опыт вывода изображения на монитор через VGA интерфейс.
	Был изучен способ загрузки программы на плату при помощи EPCS.
	Итоговый проект был собран и загружен на плату.

	
	\newpage 
	\renewcommand{\refname}{{\normalsize Список использованных источников}} 
	\centering 
	\begin{thebibliography}{9} 
		\addcontentsline{toc}{section}{\refname} 
		\bibitem{Harris} Хэррис Д. М., Хэррис С. Л. Цифровая схемотехника и архитектура компьютера. – 2015.
	\end{thebibliography}
	
\end{document} % конец документа
