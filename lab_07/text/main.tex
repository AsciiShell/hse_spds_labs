%!TEX TS-program = xelatex

% Шаблон документа LaTeX создан в 2018 году
% Алексеем Подчезерцевым
% В качестве исходных использованы шаблоны
% 	Данилом Фёдоровых (danil@fedorovykh.ru) 
%		https://www.writelatex.com/coursera/latex/5.2.2
%	LaTeX-шаблон для русской кандидатской диссертации и её автореферата.
%		https://github.com/AndreyAkinshin/Russian-Phd-LaTeX-Dissertation-Template

\documentclass[a4paper,14pt]{article}

\input{data/preambular.tex}
\begin{document} % конец преамбулы, начало документа
	\input{data/title.tex}
	\tableofcontents
	\pagebreak
	\section{Задание}
	
	\begin{enumerate}
		
		\item	Изучить раздел 5.1 DE1-SoC Factory Configuration в \_DE1-SoC\_User\_manual.pdf.
		
		\item	Модифицировать файл SPDS\_Lab\_7\_DE1\_SoC\_Default\\demo\_batch\\test.bat под выполение на вашем компьютере. 
		
		\item	Запустить проект и продемонстрировать его работу. 
			
		\item	Отобразить в отчете, каким образом генерируется изображение и передается на экран, в каком формате хранится рисунок.
		
		\item	Отобразить в отчете, как происходит генерация звука в примере проекта.
		
		\item	Отразить в отчете ответы на следующие вопросы: Что такое EPCS и для чего оно нужно? В чем отличия между форматами файлов *.jic и *.sof?
		
		\item	Вывести на 7-сегментный индикатор первые буквы имен участников бригады и номер бригады. Реализовать индикацию действий на светодиодной полосе. 

		\item	Модифицировать пример так, чтобы выводилось другое или существенно измененное изображение (изменен не только цвет). 
		
		
	\end{enumerate}

	%{\small \VerbatimInput{../03_syn_pow_5_single_cycle_always/pow_5_single_cycle_always.v}}
	
	\section{Выполнение работы}
	
	\subsection{Генерация изображения}
	
		Изображение хранится в виде 2 файлов.
		Первый файл $index\_logo.mif$ -- таблица цветов, второй $img\_data\_logo.mif$ таблица индексов.
		Индексы во втором файле это адреса в первом файле, где берется цвет пикселя для 3 цветовых каналов.
		В ПЛИС эти файлы загружаются при помощи ROM памяти.
		
		Данные изображения представляет из себя один сплошной поток адресов, чтобы разделить этот поток на строки используется $video_sync_generator$.
		В этом модуле в нужные моменты подаются сигналы о конце строки и конце изображения.
		
		После получения цвета пикселя в виде 24х битного слова оно делится на 3 канала и эти 3 значения отправляются на VGA интерфейс.
		
	\subsection{Генерация звука}
	
		Генерацию звука можно разделить на 2 этапа.
		На первом этапе задается частота на которой будет воспроизводиться звук ($VGA\_Audio\_0002.v$).
		Во время второго этапа задается амплитуда для поступающей частоты, на первый взгляд амплитуда сигнала похожа на синусоиду ($AUDIO\_DAC.v$).
		
		Принцип работы следующий, пин на который с определенной частотой поступает сигал с первого этапа. 
		Далее на этой частоте сигнал попадает на второй этап, где получает амплитуду.
		После этого сигнал отправляется в кодек, откуда и поступает на выход.
		
	\subsection{EPCS, sof, jic}
		
		EPCS - это ПЗУ на плате для конфигурации системы.
		Ядро устройства EPC разделено на два основных блока -- контроллер конфигурации и флэш-память. 
		Флэш - память используется для хранения конфигурационных данных для систем.
		Неиспользуемые части флэш-памяти можно использовать для хранения кода процессора или данных, доступ к которым можно получить с помощью внешнего интерфейса флэш-памяти после завершения конфигурации ПЛИС.
		
		Отличие jic файла от sof файла заключается в том, что jic фалом программируют через EPCS.  
	

	\section{Самостоятельная работа}
	
		Вывести на 7-сегментный индикатор первые буквы имен участников бригады и номер бригады. Реализовать индикацию действий на светодиодной полосе. 
	
		При нажатии key[0] на светодиодную ленту будут выводиться значения с кнопок.		
		При нажатии key[3] на семи сегментных индикаторах будут появится надпись: "Б5САПА".
		
		Модифицировать пример так, чтобы выводилось другое или существенно измененное изображение (изменен не только цвет). 
		
		Была скачана картинка того же размера, что и в примере.
		Далее при помощи python был создан файл с адресами цветов, аналогичный $img\_data\_logo.mif$.
		Цвета выбирались наиболее подходящие к имеющимся в наборе.
		

	\section{Выводы по работе}
	
	В ходе работы было получено представление о звуке и изображении в цифровых устройствах.
	Был получен опыт в написании и анализе кода на языке verilog.
	Был получен опыт генерации звука при помощи ПЛИС и языка verilog.
	Было изучено хранение изображений в памяти ПЛИС.
	Был получен опыт вывода изображения на монитор через VGA интерфейс.
	Был изучен способ загрузки программы на плату при помощи EPCS.
	Итоговый проект был собран и загружен на плату.

	
	\newpage 
	\renewcommand{\refname}{{\normalsize Список использованных источников}} 
	\centering 
	\begin{thebibliography}{9} 
		\addcontentsline{toc}{section}{\refname} 
		\bibitem{Harris} Хэррис Д. М., Хэррис С. Л. Цифровая схемотехника и архитектура компьютера. – 2015.
	\end{thebibliography}
	
\end{document} % конец документа
