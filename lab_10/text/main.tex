%!TEX TS-program = xelatex

% Шаблон документа LaTeX создан в 2018 году
% Алексеем Подчезерцевым
% В качестве исходных использованы шаблоны
% 	Данилом Фёдоровых (danil@fedorovykh.ru) 
%		https://www.writelatex.com/coursera/latex/5.2.2
%	LaTeX-шаблон для русской кандидатской диссертации и её автореферата.
%		https://github.com/AndreyAkinshin/Russian-Phd-LaTeX-Dissertation-Template

\documentclass[a4paper,14pt]{article}

\input{data/preambular.tex}
\begin{document} % конец преамбулы, начало документа
	\input{data/title.tex}
	\tableofcontents
	\pagebreak
	\section{Задание}
	
	\begin{enumerate}
		
		\item Изучить раздел 6.2 Users LED and KEY из \_DE1-Nano\_User\_manual.pdf;
		
		\item Запустиь проект hps\_gpio;
		
		\item Изучить мануал hps\_gpio.txt;
		
		\item Изучить раздел 7 из \_DE1-Nano\_User\_manual.pdf;
		
		\item Запустить проекты Soc\_GHRD и HPS\_FPGA\_LED;
		
		\item Реализовать циклическое переключение светодиода;
		
		\item Разработать программу, выводящую информацию об изменении состояния кнопок или переключателей;
		
		\item К проекту из п.7 подключить четырехразрядный семисегментный индикатор. модифицировать проекты для ПЛИС и HPS и вывести на индикатор текущее время.
		
		\item К проекту из п.7: реализовать для данного индикатора бегущую строку для вывода информации. Вывести на индикатор текущую дату (день, месяц, год)
		
	\end{enumerate}
	\pagebreak
	%{\small \VerbatimInput{../03_syn_pow_5_single_cycle_always/pow_5_single_cycle_always.v}}
	
	\section{Выполнение работы}
	
	Был изучен раздел раздел 6.2 из \_DE1-Nano\_User\_manual.pdf. Схема работы платы представлена на 	рис. \ref{fig:gpiodemonstrtion}.
	
	\begin{figure}[H]
		\centering
		\includegraphics[width=0.7\linewidth]{images/gpio_demonstrtion}
		\caption{Демонстрация GPIO}
		\label{fig:gpiodemonstrtion}
	\end{figure}

	Был запущен проект hps\_gpio, в котором по нажатию кнопки гаснет и загорается светодиод.
	
	В мануале hps\_gpio.txt говорится о том, как можно узнать общую информацию о GPI и о том как можно смотреть состояние GPIO через командную строку.
	
	Раздел 7 из \_DE1-Nano\_User\_manual.pdf был изучен. Дигаграмма подключения устройств представлена на рис. \ref{fig:axi-bridge-block-diagram}.
	
	\begin{figure}[H]
		\centering
		\includegraphics[width=0.7\linewidth]{"images/AXI Bridge Block Diagram"}
		\caption{AXI Bridge Block Diagram}
		\label{fig:axi-bridge-block-diagram}
	\end{figure}
	
	

	\section{Самостоятельная работа}
	
	Все проекты были собраны загружены на плату:
	
	\begin{enumerate}
		\item Реализовать циклическое переключение светодиода;
		
		\item Разработать программу, выводящую информацию об изменении состояния кнопок или переключателей;
		
		\item К проекту из п.7 подключить четырехразрядный семисегментный индикатор. модифицировать проекты для ПЛИС и HPS и вывести на индикатор текущее время.
		
		\item К проекту из п.7: реализовать для данного индикатора бегущую строку для вывода информации. Вывести на индикатор текущую дату (день, месяц, год)
	\end{enumerate}
	
	Ссылка на репозиторий с программами: 
	
	\href{https://github.com/AsciiShell/hse_spds_labs/tree/master/lab_10}{https://github.com/AsciiShell/hse\_spds\_labs/tree/master/lab\_10	}
	
	
	\section{Выводы по работе}
	
	В ходе работы был изучен принцип загрзки ОС Linux на плту.
	Был рассмотрен вариант взаимодействия с периферийными устройствами через HPS.
	Был получен опыт в написании и анализе кода на языке С.
	Был получен опыт в компиляции программ для ПЛИС на языке C.
	Был получен опыт в загрузке программы и работе с ОС на ПЛИС через PuTTY.	
	Был повторен метод создания программ для ПЛИС при помощи QSYS.	
	Итоговые проекты были собраны и загружены на плату.
	
	\newpage 
	\renewcommand{\refname}{{\normalsize Список использованных источников}} 
	\centering 
	\begin{thebibliography}{9} 
		\addcontentsline{toc}{section}{\refname} 
		\bibitem{Harris} Хэррис Д. М., Хэррис С. Л. Цифровая схемотехника и архитектура компьютера. – 2015.
	\end{thebibliography}
	
\end{document} % конец документа
