%!TEX TS-program = xelatex

% Шаблон документа LaTeX создан в 2018 году
% Алексеем Подчезерцевым
% В качестве исходных использованы шаблоны
% 	Данилом Фёдоровых (danil@fedorovykh.ru) 
%		https://www.writelatex.com/coursera/latex/5.2.2
%	LaTeX-шаблон для русской кандидатской диссертации и её автореферата.
%		https://github.com/AndreyAkinshin/Russian-Phd-LaTeX-Dissertation-Template

\documentclass[a4paper,14pt]{article}

\input{data/preambular.tex}
\begin{document} % конец преамбулы, начало документа
	\input{data/title.tex}
	\tableofcontents
	\pagebreak
	\section{Задание}
	
	\begin{enumerate}
		\item В соответствии с мануалом Basic\_DSP\_Manual.pdf и используя файлы из /SPDS\_Lab\_5\_dop\_materials 1 и 2, создать проект, в котором сигнал с линейного входа от микрофона (или плеера) подается на линейный выход (наушники, динамики).
		
		\item Добавить управление от кнопки, убирающее один из каналов или меняющее каналы местами.
		
		\item Разработать генератор шума и добавить его к выходному звуку.
		
		\item Провести моделирование проекта и провести анализ полученных вейвформ.
		
		\item Выполнить прототипирование.
		
		\item Используя делитель частоты из ЛР1 (опционально, можно использовать PLL и самописные делители частоты) реализовать индикацию на светодиодной ленте и 7-сегментном индикаторе.
		
		\item Разработать диктофон (записывает звук, пока нажата кнопка, при нажатии другой кнопки – воспроизводит запись циклически).
		
		
	\end{enumerate}

	%{\small \VerbatimInput{../03_syn_pow_5_single_cycle_always/pow_5_single_cycle_always.v}}
	
	\section{Выполнение работы}
	
		Был создан модуль, который перенаправляет считанные значения из кодека в кодек для их воспроизведения, учитывая при этом готовность кодека к отдаче и приему сигналов.
		Левая дорожка визуализируется на светодиодной ленте, а правая на восьмисегментых индикаторах.
		Помимо этого при нажатии кнопки (KEY[1]) каналы меняются местами.
		Листинг и вейвформа (рис. \ref{fig:screenshot93}) приведены ниже:
		
		{\small \VerbatimInput{../SPDS_Lab_5_dop_materials/z2.v}}
		
		
		\begin{figure}[H]
			\centering
			\includegraphics[width=0.7\linewidth]{images/Screenshot_93}
			\caption{Вейвформа при которой дорожки меняются местами}
			\label{fig:screenshot93}
		\end{figure}
	
		Далее был создан генератор шума.
		Листинг и вейвформа (рис. \ref{fig:screenshot94}) приведены ниже:
		
		{\small \VerbatimInput{../SPDS_Lab_5_dop_materials/z3.v}}
		
		\begin{figure}[H]
			\centering
			\includegraphics[width=0.7\linewidth]{images/Screenshot_94}
			\caption{Вейвформа с генератором шума}
			\label{fig:screenshot94}
		\end{figure}

	\section{Самостоятельная работа}
	
		Далее был спроектирован диктофон. 
		
		\begin{itemize}
			\item KEY[0] - сброс памяти;
			
			\item KEY[2] - запись;
						
			\item KEY[0] - воспроизведение.
		\end{itemize}
	
		Листинг и вейвформа (рис. \ref{fig:z5}) приведены ниже:
		
		{\small \VerbatimInput{../SPDS_Lab_5_dop_materials/z5.v}}
		
		\begin{figure}[H]
			\centering
			\includegraphics[width=0.7\linewidth]{images/z5}
			\caption{Вейвформа с диктофоном}
			\label{fig:z5}
		\end{figure}

	Модул памяти был спроектирован самостоятельно, листинг ниже:
	
	Листинг и вейвформа (рис. \ref{fig:z5}) приведены ниже:
	
	{\small \VerbatimInput{../proj/my_mem.v}}

	\section{Выводы по работе}
	
	В ходе работы было получено базовое представление звука в цифровых устройствах.
	Был получен опыт в написании кода на языке verilog.
	Был получен опыт обработки звука при помощи ПЛИС и языка verilog.
	Были изучены звуковые фильтры.
	Была исследована возможность добавления шумов к аудио дорожке.
	Были построены и проанализированы вейвформы для спроектированных схем.
	Итоговый проект был собран и загружен на плату.

	
	\newpage 
	\renewcommand{\refname}{{\normalsize Список использованных источников}} 
	\centering 
	\begin{thebibliography}{9} 
		\addcontentsline{toc}{section}{\refname} 
		\bibitem{Harris} Хэррис Д. М., Хэррис С. Л. Цифровая схемотехника и архитектура компьютера. – 2015.
	\end{thebibliography}
	
\end{document} % конец документа
