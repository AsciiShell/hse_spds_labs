%!TEX TS-program = xelatex

% Шаблон документа LaTeX создан в 2018 году
% Алексеем Подчезерцевым
% В качестве исходных использованы шаблоны
% 	Данилом Фёдоровых (danil@fedorovykh.ru) 
%		https://www.writelatex.com/coursera/latex/5.2.2
%	LaTeX-шаблон для русской кандидатской диссертации и её автореферата.
%		https://github.com/AndreyAkinshin/Russian-Phd-LaTeX-Dissertation-Template

\documentclass[a4paper,14pt]{article}

\input{data/preambular.tex}
\begin{document} % конец преамбулы, начало документа
	\input{data/title.tex}
	\tableofcontents
	\pagebreak
	\section{Задание}
	
	\begin{enumerate}
		\item Создать проект и настроить его под плату.
		
		\item Реализовать программы из Практической работы 1 00\_counter/, 01\_fibonacci/ и 02\_sqrt/ на процессоре Nios II.
		
		\item Найти сумму геометрической прогрессии (количество членов прогрессии = ваш вариант \% 25 + 5, знаменатель прогрессии = ваш вариант \% 8 + 3).
		
		\item Переделать пример из Этапа 1, так, чтобы для программ из Практической работы 1 00\_counter/, 01\_fibonacci/ и 02\_sqrt/ на процессоре Nios II.
		
		\item Переставить в исходном числе двойки битов в обратном порядке, т.е. номера битов в исходных данных были такие – 76543210, а стали такими – 01237654.
		
	\end{enumerate}

	%{\small \VerbatimInput{../03_syn_pow_5_single_cycle_always/pow_5_single_cycle_always.v}}
	
	\section{Выполнение работы}
	
	\subsection{Этап 1}
	
	Согласно мануалу был создан проект, на рис. \ref{fig:scheme} приведено использование процессора Nios2 в $Platform designer$.
	К процессору подсоединена память и два модуля ввода/вывода информации, а также модуль генерации тактовых импульсов.
	
	Проект из $Platform designer$ был сгенерирован, полученный модуль подключен в среде Quartusи и скомпилирован.
	
	Далее используя $Intel FPGA Monitor Program 17.1$ загружаем необходимую программу и конфигурацию платы на плату.
	Текст программы:
	
	{\small \VerbatimInput{../programs/lights.c}}
	
	\begin{figure}[h]
		\centering
		\includegraphics[width=0.9\linewidth]{images/scheme}
		\caption{Проект в $Platform designer$ этап 1}
		\label{fig:scheme}
	\end{figure}
	
	\begin{enumerate}
		\item Для чего нужны регистры, и что будет, если изменить значение регистра r с номером = ваш вариант \% 3 + 2.
		
		Регистры нужны для хранения информации.
		Если изменить значение в 4 регистре, то вскоре программа перезапишет значение в этом регистре на значение, которое пришло изи процессора.
		Если успеть изменить значение в этом регистре сразу после перезаписи процессора, то введенное значение окажется на светодиодной ленте.
		
		\item Что находится в закладке Memory? Модифицируйте значение по адресу 0x00002000 на значение 0x003FFD06. К чему это привело, и почему?
		
		В этой вкладке отображается RAM.
		Из ячейки по адресу 0x00002000 значение передается на светодиодную ленту.
		Модификация значения по адресу 0x00002000 приведет к тому что со следующим тактовым импульсом записанное значение передастся на светодиодную ленту.
	\end{enumerate}

	\subsection{Этап 2}

	Согласно мануалу был создан проект, на рис. \ref{fig:scheme} приведено использование процессора Nios2 в $Platform designer$.
	К процессору подсоединена память и свой модуль вывода информации, а также модуль генерации тактовых импульсов.
	
	Добавление своих компонентов сильно расширяет возможности разработчика.
	Можно создавать свои модули, которых может не быть в библиотеке, а также расширять возможности имеющихся.
	В данном проекте выводится полноценное слово (32 бита) в отличие от предыдущего этапа, где выводилось только 8 битов.
	Также в этот раз уменьшена задержка при отображении результатов.
	
	\begin{figure}[h]
		\centering
		\includegraphics[width=0.9\linewidth]{images/scheme1}
		\caption{Проект в $Platform designer$ этап 2}
		\label{fig:scheme1}
	\end{figure}
	

	\section{Самостоятельная работа}
	
	\subsection{Этап 1}
	
	Далее были подготовлены программ 00\_counter/, 01\_fibonacci/ и 02\_sqrt/ для процессора процессоре Nios II.
	
	00\_counter:
	
	{\small \VerbatimInput{../programs/counter_n.s}}
	
	01\_fibonacci:
	
	{\small \VerbatimInput{../programs/fibonacci_n.s}}
	
	02\_sqrt:
	
	{\small \VerbatimInput{../programs/sqrt_n.s}}
	
	Найти сумму геометрической прогрессии (количество членов прогрессии = 10, знаменатель прогрессии = 8).
	
	Код:
	
	{\small \VerbatimInput{../programs/sum_n.s}}
	
	\subsection{Этап 2}
	
	Программы 00\_counter/, 01\_fibonacci/ и 02\_sqrt/ отличаются от предыдущего этапа тем, что у них не задаются значения через свичи и изменен адрес по-которому записывается результат.
	В остальном программы работают точно также.
	
	Переставить в исходном числе двойки битов в обратном порядке, т.е. номера битов в исходных данных были такие – 76543210, а стали такими – 01237654.
	
	Код программы:
	
	{\small \VerbatimInput{../programs/transfer.s}}

	\section{Выводы по работе}
	
	В ходе работы был получен опыт работы в Platform designer.
	Был получен опыт в создании своих компонентов в Platform designer.
	Был изучен порядок работы с памятью и процессором NIOS II.
	Было изучено взаимодействие проекта в с элементами ввода/вывода на плате.
	Был получен опыт в написании кода на ассемблере для процессора NIOS II.
	Был получен опыт в написании кода на языке verilog.
	Итоговый проект был собран и загружен на плату при помощи программы Intel FPGA Monitor Program.

	
	\newpage 
	\renewcommand{\refname}{{\normalsize Список использованных источников}} 
	\centering 
	\begin{thebibliography}{9} 
		\addcontentsline{toc}{section}{\refname} 
		\bibitem{Harris} Хэррис Д. М., Хэррис С. Л. Цифровая схемотехника и архитектура компьютера. – 2015.
	\end{thebibliography}
	
\end{document} % конец документа
