%!TEX TS-program = xelatex

% Шаблон документа LaTeX создан в 2018 году
% Алексеем Подчезерцевым
% В качестве исходных использованы шаблоны
% 	Данилом Фёдоровых (danil@fedorovykh.ru) 
%		https://www.writelatex.com/coursera/latex/5.2.2
%	LaTeX-шаблон для русской кандидатской диссертации и её автореферата.
%		https://github.com/AndreyAkinshin/Russian-Phd-LaTeX-Dissertation-Template

\documentclass[a4paper,14pt]{article}

\input{data/preambular.tex}
\begin{document} % конец преамбулы, начало документа
	\input{data/title.tex}
	\tableofcontents
	\pagebreak
	\section{Задание}
	
	\begin{enumerate}
		\item Изучить разделы 6.2, 6.3, 6.4, 6.7 и приложение В книги H\&H.
		Добавить в микропроцессор в соответствии со своим вариантом	поддержку следующих команд: j, xori, sllv, nor.
		
		\item Разработать программу, продемонстрировать на модели и прототипе правильность их	работы.
		Разработать в соответствии со своим вариантом программу, продемонстрировать на модели и прототипе правильность ее работы.
		Добавить ее в проект микропроцессора; добавить в папку с программой файл описания.
		Использовать только те команды, которые есть в процессоре.
		
		6) Найти сумму геометрической прогрессии (количество членов прогрессии = ваш
		вариант \% 30 + 3, знаменатель прогрессии = ваш вариант \% 10 + 1)
		
		\item Перейти в ветку проекта schoolMIPS 01\_mmio.
		Скачать новую версию процессора и выполнить на вашей плате (или DE10-Lite)
		программы 00\_counter, 01\_fibonacci, 02\_sqrt. Убедиться, что они работают также.
		Выполнить одну из программ (по вариантам):
		
		2) 04\_gpio
	\end{enumerate}

	%{\small \VerbatimInput{../03_syn_pow_5_single_cycle_always/pow_5_single_cycle_always.v}}
	
	\section{Выполнение работы}
	
	\subsection{Моделирование счетчика}
	
	Ассемблерный код счетчика представлен на листинге ниже.
	
	{\small \VerbatimInput{../program/00_counter/main.S}}
	
	Моделирование программы проводилось в среде Icarus Verilog.
	
	Ниже приведена часть логов из выполнения программы:
	
	{\small \VerbatimInput{./logs/00_logs.txt}}
	
	Вейвформа при моделировании программы (рис. \ref{fig:00wvf}).
	
	\begin{figure}[H]
		\centering
		\includegraphics[width=0.95\linewidth]{images/00_wvf}
		\caption{Вейвформа для программы счетчика}
		\label{fig:00wvf}
	\end{figure}

	Можно заметить, что значение интересующего нас регистра постепенно увеличивается на 1.
	
	
	\subsection{Моделирование последовательности Фибоначчи}
	
	Ассемблерный код функции для подсчета значений последовательности Фибоначчи представлен на листинге ниже.
	
	{\small \VerbatimInput{../program/01_fibonacci/main.S}}
	
	Моделирование программы проводилось в среде Icarus Verilog.
	
	Ниже приведена часть логов из выполнения программы:
	
	{\small \VerbatimInput{./logs/01_logs.txt}}
	
	Вейвформа при моделировании программы (рис. \ref{fig:01wvf}).
	
	\begin{figure}[H]
		\centering
		\includegraphics[width=0.95\linewidth]{images/01_wvf}
		\caption{Вейвформа для последовательности Фибоначчи}
		\label{fig:01wvf}
	\end{figure}

	
	

	\subsection{Моделирование извлечения квадратного корня}
	
	Ассемблерный код функции для вычисления квадратного корня представлен на листинге ниже.
	
	{\small \VerbatimInput{../program/02_sqrt/main.S}}
	
	Моделирование программы проводилось в среде Icarus Verilog.
	
	Лог выполнения программы:
	
	{\small \VerbatimInput{./logs/02_sqrt_logs.txt}}
	
	Вейвформа при моделировании программы (рис. \ref{fig:02wvf}).
	
	\begin{figure}[H]
		\centering
		\includegraphics[width=0.95\linewidth]{images/02_wvf}
		\caption{Вейвформа для программы извлечения квадратного корня}
		\label{fig:02wvf}
	\end{figure}
	
	
	\section{Самостоятельная работа}
	
	\section{Выводы по работе}
	
	В ходе работы получен опыт проектирования схем в программе Quartus с помощью языка Verilog.
	Полученное устройство было протестировано с помощью бенчтестов в программе Quartus Simulation Waveform editor.
	В процессе работы были смоделированы устройства для конвейерной обработки данных и изучены различные способы моделирования.
	В процессе был получен опыт работы с платой DE10-Lite, на которой проверялась работоспособность полученного устройства.
	
	\newpage 
	\renewcommand{\refname}{{\normalsize Список использованных источников}} 
	\centering 
	\begin{thebibliography}{9} 
		\addcontentsline{toc}{section}{\refname} 
		\bibitem{Verilog} Thomas D., Moorby P. The Verilog Hardware Description Language. – Springer Science \& Business Media, 2008.
		\bibitem{citekey} Khor W. Y. et al. Evaluation of FPGA Based QSPI Flash Access Using Partial Reconfiguration //2019 7th International Conference on Smart Computing \& Communications (ICSCC). – IEEE, 2019. – С. 1-5
	\end{thebibliography}
	
\end{document} % конец документа
