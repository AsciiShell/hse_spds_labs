%!TEX TS-program = xelatex

% Шаблон документа LaTeX создан в 2018 году
% Алексеем Подчезерцевым
% В качестве исходных использованы шаблоны
% 	Данилом Фёдоровых (danil@fedorovykh.ru) 
%		https://www.writelatex.com/coursera/latex/5.2.2
%	LaTeX-шаблон для русской кандидатской диссертации и её автореферата.
%		https://github.com/AndreyAkinshin/Russian-Phd-LaTeX-Dissertation-Template

\documentclass[a4paper,14pt]{article}

\input{data/preambular.tex}
\begin{document} % конец преамбулы, начало документа
    \input{data/title.tex}
    \tableofcontents
    \pagebreak


    \section{Задание}

    \begin{enumerate}
        \item В соответствии с мануалом \_SPDS\_Lab\_2\_GPIO.pdf добавить на плату расширение в виде 7-ми сегментного индикатора (обязательно), DIP переключателя (обязательно) и светодиодов (если требуется).

        \item Изменить настройки проекта и вывести на новую дополнительную периферию информация о результатах работы процессора, провести тестирование работы программ, разработанных в работе 1.

        \item Создать копию проекта и добавить в процессор дополнительный неархитектурный 8ми битный вход (к этому входу в файле верхнего уровня иерархии подключить DIP переключатель). Добавить в систему команд процессора ноовые команды, которые будут загружать данные с нового входа в заданный регистр и выставлять результат на другом выходе. В соответствии с вариантом (ваш вариант \% 5 + 1):

        Реализовать дополнение байта до 32-х бит "0"-ми.

        Переработать программы 01\_fibonacci/ и 02\_sqrt/, чтобы загружать значения из вне и выводить результат на внешнюю периферию платы. Провести моделирование и тестирование работоспособности проекта на прототипе.

        \item Скачать новую версию процессора и выполнить на вашей плате программы из предыдущих лабораторных работ. Убедиться, что они работают как в предыдущей работе. Выполнить программы: 04\_irq\_timer и 05\_exc\_ri. Добавить подробные комментарии к программам. Объяснить, чем эта версия schoolMIPS отличается от базовой.
    \end{enumerate}

    %{\small \VerbatimInput{../03_syn_pow_5_single_cycle_always/pow_5_single_cycle_always.v}}


    \section{Выполнение работы}

    %{\small \VerbatimInput{../program/00_counter/main.S}}

    \subsection{Изменение архитектуры процессора}

    Для поддержки процессором дополнительной переферии были внесены следующие изменения:

    \begin{itemize}

        \item Добавлен модуль $sm\_hex\_display\_digit$

        \item Добавлен модуль $sm\_hex\_display\_our$

        \item Изменен модуль верхнего уровня $de0\_cv$

    \end{itemize}

    Модуль $sm\_hex\_display\_digit$ поочередно через 11 пинов GPIO отображает результат на тройном 8-ми сегментном индикаторе.
    Листинг представлен ниже:

    {\small \VerbatimInput{../src/sm_hex_display_digit.v}}

    Модуль $sm\_hex\_display\_our$ декодирует символ в его представление для отображения на 8-ми сегментном индикаторе.
    Листинг представлен ниже:

    {\small \VerbatimInput{logs/sm_hex_display_our.v}}

    В модуле верхнего уровня $de0\_cv$ часть пинов из GPIO\_0 связывается с gpioInput, GPIO\_1 связывается с gpioOutput.

    {\small \VerbatimInput{logs/changes_de0_cv.txt}}

    \subsection{Написание программ}

    GPIO input и GPIO output настроены таким образом, что для загрузки и выгрузки значений не нужны дополнительные инструкции.
    Для этих задач подходят уже имеющиеся в нашем процессоре инструкции $lw$ и $sw$.

    \subsubsection{последовательность Фибоначчи}

    Код программы вычисления последовательности Фибоначчи:

    {\small \VerbatimInput{../program/91_fibo/main.S}}

    Вейвформы:

    \begin{figure}[H]
        \centering
        \includegraphics[width=0.7\linewidth]{images/fibo_wvf}
        \caption{Вейвформа моделирования вычисления последовательность Фибоначчи}
        \label{fig:fibowvf}
    \end{figure}

    \subsubsection{Вычисление квадратного корня}

    Код программы вычисления квадратного корня:

    {\small \VerbatimInput{../program/92_sqrt/main.S}}

    Вейвформы:

    \begin{figure}[H]
        \centering
        \includegraphics[width=0.7\linewidth]{images/sqrt_wvf}
        \caption{Вейвформа моделирования вычисления квадратного корня}
        \label{fig:sqrtwvf}
    \end{figure}

	\subsection{Моделирование программ на процессоре из ветки 02\_irq}

	Было произведено моделирование программ  00\_counter, 01\_fibonacci, 02\_sqrt, все по-прежнему работают, но никаких изменений в логах или вейвформах нет.
	
%   В соответствии с вариантом была смоделирована программа 04\_gpio. Ее ассемблерный код:
	
%	{\small \VerbatimInput{../program/04_gpio/main.S}}
	
%	В метке init происходит обнуление регистра t0 и запись в регистр t1 длительности задержки.
%	Далее в метке delay циклически увеличивается регистр t0, пока он не станет равен t1.
%	После этого регистр t0 обнуляется.
%	Под меткой read происходит чтение данных из памяти для GPIO.
%	Под меткой write происходит запись данных в память для GPIO.
%	далее возвращаемся к метке delay.
	
%	Ниже приведена часть логов из выполнения программы:
	
%	{\small \VerbatimInput{./logs/04_gpio_logs.txt}}
	
%	Вейвформа при моделировании программы (рис. \ref{fig:04wvf}).
	
%	\begin{figure}[H]
%		\centering
%		\includegraphics[width=0.95\linewidth]{images/04_wvf}
%		\caption{Вейвформа для GPIO}
%		\label{fig:04wvf}
%	\end{figure}

    \section{Выводы по работе}

    В ходе работы получен опыт в написании кода на ассемблере MIPS.
    Был получен опыт чтения машинного кода.
    Был смоделирован процессор MIPS со входами GPIO.
    Был получен опыт работы с дополнительной переферией через GPIO.
    Были переделаны программы для демонстрации возможностей работы с GPIO.
    Было произведено моделирование в программе Icarus Verilog.
    Итоговый проект был собран и загружен на плату.

    \newpage
    \renewcommand{\refname}{{\normalsize Список использованных источников}}
    \centering
    \begin{thebibliography}{9}
        \addcontentsline{toc}{section}{\refname}
        \bibitem{Harris} Хэррис Д. М., Хэррис С. Л. Цифровая схемотехника и архитектура компьютера. – 2015.
    \end{thebibliography}

\end{document} % конец документа
