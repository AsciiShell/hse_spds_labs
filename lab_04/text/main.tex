%!TEX TS-program = xelatex

% Шаблон документа LaTeX создан в 2018 году
% Алексеем Подчезерцевым
% В качестве исходных использованы шаблоны
% 	Данилом Фёдоровых (danil@fedorovykh.ru) 
%		https://www.writelatex.com/coursera/latex/5.2.2
%	LaTeX-шаблон для русской кандидатской диссертации и её автореферата.
%		https://github.com/AndreyAkinshin/Russian-Phd-LaTeX-Dissertation-Template

\documentclass[a4paper,14pt]{article}

\input{data/preambular.tex}
\begin{document} % конец преамбулы, начало документа
	\input{data/title.tex}
	\tableofcontents
	\pagebreak
	\section{Задание}
	
	\begin{enumerate}
		\item В соответствии с мануалом Altera\_QurtusII\_SignalTap\_manual.pdf создать простой проект, где на светодиоды подаются комбинации нажатия на кнопки по принципу дешифратора.
		
		\item Скомпилировать проект и выполнить прототипирование, убедиться в корректности работы проекта.
		
		\item Включить Quartus II TalkBack Feature.
		
		\item Запустить SignalTap II и в соответствии с мануалом настроить отслеживание нажатия следующей комбинации кнопок = ваш вариан\% 4 +1.
		
		\item Привязать тактовый сигнал.
		
		\item Выполнить компиляцию проекта. Запрограммировать плату. Отследить событие нажатия комбинации кнопок. Привести результаты в отчете.
		
		\item Отследить события прихода фронтов сигналов для комбинации кнопок = (ваш вариант+1)\% 4 +1. Продемонстрировать отличия от отслеживания события по уровню.
		
		\item Добавить собственные условия для срабатывания событий в режиме Advanced.
			
		\item Настроить множественное отслеживание событий используя Sample Depth and Buffer Acquisition Modes.
		
		\item Проиллюстрировать на своем проекте использование Synthesis Keep Directive и необходимость ее использования.
		
		\item Оформить отчет.
		
	\end{enumerate}

	%{\small \VerbatimInput{../03_syn_pow_5_single_cycle_always/pow_5_single_cycle_always.v}}
	
	\section{Выполнение работы}
	
	Был создан проект, который представляет из себя простейший дешифратор.
	Код дешифратора:
	
	{\small \VerbatimInput{../first_part/de0_cv/project/keys.v}}
	
	Проект скомпилировался, его RTL представление (рис. \ref{fig:rtl}) и вейвформа (рис. \ref{fig:wvf}).
	
	\begin{figure}[H]
		\centering
		\includegraphics[width=0.6\linewidth]{images/RTL}
		\caption{RTL представление проекта}
		\label{fig:rtl}
	\end{figure}

	\begin{figure}[H]
		\centering
		\includegraphics[width=0.6\linewidth]{images/WVF}
		\caption{Моделирование проекта}
		\label{fig:wvf}
	\end{figure}
	
	Отслеживание нужной комбинации проводилось при помощи $SignalTap$, таблица $setup$ представлена на рис. \ref{fig:setupkey1}.
	
	\begin{figure}[H]
		\centering
		\includegraphics[width=0.7\linewidth]{images/setup_key_1}
		\caption{Таблица setup для отслеживания нужной комбинации}
		\label{fig:setupkey1}
	\end{figure}

	
	Отслеживание приходов фронтов проводилось в режиме $advanced$.
	Т. к. одновременный приход двух фронтов на частоте 50 МГц почти невозможен, будем отслеживать интересующую нас комбинацию по последнему фронту, т. е. один из ключей уже должен быть в нужном положении.
	Схема в $advanced$ режиме представлена на рис. \ref{fig:advancedscheme}.
	

	\begin{figure}[H]
		\centering
		\includegraphics[width=0.7\linewidth]{images/advanced_scheme}
		\caption{}
		\label{fig:advancedscheme}
	\end{figure}
	


	
	
	\section{Самостоятельная работа}
	
	

	\subsection{Моделирование программ на процессоре из ветки 01\_mmio}
	
	Было произведено моделирование программ  00\_counter, 01\_fibonacci, 02\_sqrt, все по-прежнему работают, но никаких изменений в логах или вейвформах нет.
	
	В соответствии с вариантом была смоделирована программа 04\_gpio. Ее ассемблерный код:
	
	{\small \VerbatimInput{../program/04_gpio/main.S}}
	
	В метке init происходит обнуление регистра t0 и запись в регистр t1 длительности задержки.
	Далее в метке delay циклически увеличивается регистр t0, пока он не станет равен t1.
	После этого регистр t0 обнуляется.
	Под меткой read происходит чтение данных из памяти для GPIO.
	Под меткой write происходит запись данных в память для GPIO.
	далее возвращаемся к метке delay.
	
	Ниже приведена часть логов из выполнения программы:
	
	{\small \VerbatimInput{./logs/04_gpio_logs.txt}}
	
	Вейвформа при моделировании программы (рис. \ref{fig:04wvf}).
	
	\begin{figure}[H]
		\centering
		\includegraphics[width=0.95\linewidth]{images/04_wvf}
		\caption{Вейвформа для GPIO}
		\label{fig:04wvf}
	\end{figure}
	
	
	
	\section{Выводы по работе}
	
	В ходе работы получен опыт в написании кода на ассемблере MIPS.
	Был получен опыт чтения машинного кода.
	Был смоделирован процессор MIPS.
	Изучены и добавлены дополнительные команды процессора MIPS.
	Была написана программа для демонстрации работы дополнительных команд.
	Было произведено моделирование в программе Icarus Verilog.
	Итоговый проект был собран и загружен на плату.
	
	\newpage 
	\renewcommand{\refname}{{\normalsize Список использованных источников}} 
	\centering 
	\begin{thebibliography}{9} 
		\addcontentsline{toc}{section}{\refname} 
		\bibitem{Harris} Хэррис Д. М., Хэррис С. Л. Цифровая схемотехника и архитектура компьютера. – 2015.
	\end{thebibliography}
	
\end{document} % конец документа
